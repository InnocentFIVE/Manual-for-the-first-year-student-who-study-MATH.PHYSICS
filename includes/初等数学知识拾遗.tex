\begin{center}
    \chapter[{初等数学知识拾遗}]{初等数学知识拾遗\\Pre-knowledge of Advanced Mathematics}\label{cp:初等数学知识拾遗}

    {\itshape \normalsize 云与星与歌}

    \pagestyle{mainlatter}
    \addauthor{云与星与歌}
\end{center}
\medskip

\normalsize
\pagestyle{mainlatter}
\section*{前言}
 {\itshape 金秋已至,又一批新生踏入了大学校园。然而,无论是数学专业,还是非数理科,抑或是工农经管,甚至某些学校的文史哲类,都开设了高等数学
  课程。对很多新生来说,“高数”这个名字是地狱般的存在:极限、连续、可导等一系列概念把大家绕得团团转,不仅如此,反三角函数、归纳法、极坐标系等
  “高中老师讲过”的初等知识也是大家的薄弱之处。这里,我们为大家总结了一些初等数学中常用,却在中学阶段被遗漏的知识和方法,以供学习和参考。}
\addtocounter{section}{-1}
\section{凡例}

\begin{center}
    \begin{tblr}{colspec = {X[-1, l]|X[l]}, width = 0.7\textwidth}\hline
        记号                                 & 说明                            \\ \hline
        $a\mid b$                            & $a$ 整除 $b$,即 $b / a$ 是整数 \\
        $\exists !$                          & 存在且唯一                      \\
        $\left\lfloor \cdots  \right\rfloor$ & 下取整函数                      \\
        \hline\end{tblr}
\end{center}

\section{三角函数知识补充}

在高中阶段,我们已经学习了三角函数的相关知识。对大家来说,分析形如$f(x)=A\sin(\omega x+b) $%
的函数的增减性并非难事,利用正弦公式和余弦公式解三角形(当然,包括其他与函数最值结合的阴间题目)也为%
大家所熟悉。然而,由于教学大纲的变化,许多重要的三角函数知识被删去,对三角恒等变换的要求降低%
,这无疑给大家适应大学数学的学习带来了不少困难。下面,我们就来补充一些常用的三角函数知识。
\subsection{和差化积与积化和差}
\label{积化和差}

我们熟知如下 cos 的两角和差公式
\begin{gather}
    \cos(\delta +\theta )=\cos \delta  \cos \theta -\sin \delta  \sin \theta,\label{AtM1}\\
    \cos (\delta -\theta )=\cos \delta  \cos \theta +\sin \delta  \sin \theta.\label{AtM2}
\end{gather}
\eqref{AtM1} \eqref{AtM2} 两式相加后,我们在等式两边除以 $2$ 之后得到
\begin{equation}
    \cos \delta  \cos \theta=\frac{1}{2}[\cos(\delta +\theta )+\cos (\delta -\theta )],
\end{equation}
如果我们用$\alpha,\beta$分别替换括号内的$\delta +\theta,\delta -\theta$, 就得到
\begin{equation}
    \cos \alpha + \cos \beta = 2\cos\left( \frac{\alpha+\beta}{2} \right) \cos\left( \frac{\alpha-\beta}{2} \right) .
\end{equation}
这就是积化和差与和差化积公式之一。用完全相同的方法,我们可以得出另外6个公式。限于篇幅,就不在此处一一列出了。

然而,我们不仅要知道公式本身,还得掌握它的用法。

\begin{example}
    在$\bigtriangleup ABC $中,求证:%求证后面用中文逗号
    $\sin A+\sin B+\sin C=4\cos \frac{A}{2}\cos \frac{B}{2}\cos \frac{C}{2}$.
\end{example}


\begin{prove}
    这是三角形中一个经典的结论,利用它我们可以解决很多问题。
    \[
        \begin{aligned}
             & \mathrel{\phantom{=}}\sin A+\sin B+\sin C                                                                                        \\
             & =\sin A+\sin B+\sin [\symup\pi -(A+B)] =\sin A+\sin B+\sin(A+B)                                                                  \\
             & =2\sin\left( \frac{A+B}{2}\right)\cos\left( \frac{A-B}{2}\right)+2\sin\left( \frac{A+B}{2}\right)\cos\left( \frac{A+B}{2}\right) \\
             & =2\sin\left( \frac{A+B}{2}\right)\left (\cos\left( \frac{A-B}{2}\right)+\cos\left( \frac{A+B}{2}\right)\right )                  \\
             & =2\sin\left( \frac{\symup\pi -C}{2} \right) \left (2\cos\frac{A}{2}\cos\frac{B}{2}\right )                                       \\
             & =4\cos \frac{A}{2}\cos \frac{B}{2}\cos \frac{C}{2}.
        \end{aligned}
    \]
\end{prove}

\begin{example}
    求证一个奇怪的等式:$\cos x=2\sin x(\sin 2x+\sin 4x+\sin 6x+\cdots)$.
\end{example}

\begin{prove}
    据积化和差公式
    \[
        2\sin nx \sin x = \cos(n+1)x-\cos(n-1)x.
    \]

    所以
    \[
        \begin{aligned}
            \cos x & =(\cos x-\cos 3x)+(\cos 3x-\cos 5x)+(\cos 5x-\cos 7x)+\cdots \\
                   & =2\sin x \sin 2x +\sin x\sin 4x+\sin x\sin 6x+\cdots         \\
                   & =2\sin x(\sin 2x+\sin 4x+\sin 6x+\cdots).
        \end{aligned}
    \]
\end{prove}


\subsection{反三角函数}
对于反三角函数,我们有定义如下表:

\begin{center}
    \begin{tblr}{colspec = {c|c|c|c}, row{2,3,4} = {abovesep=5pt,belowsep=5pt}}
        \hline 函数名 & 定义                         & 定义域       & 值域                                                         \\
        \hline
        反正弦        & 若$x=\sin y$,则$y=\arcsin x$ & $[-1,1]$     & $\left[ -\frac{\symup\pi }{2},\frac{\symup\pi }{2} \right] $ \\
        反余弦        & 若$x=\cos y$,则$y=\arccos x$ & $[-1,1]$     & $[0,\symup\pi ]$                                             \\
        反正切        & 若$x=\tan y$,则$y=\arctan x$ & $\mathbb{R}$ & $\left( -\frac{\symup\pi }{2},\frac{\symup\pi }{2} \right) $ \\
        \hline
    \end{tblr}
\end{center}

事实上,这里的值域实际上是指反三角函数的\textbf{主值}的取值范围。主值与自变量一一对应,仍然满足函数的定义。


下面我们给出一些关于反三角函数的基本恒等式(其中所有$x$的取值须满足对应函数的定义域)。
\[
    \begin{tblr}{X[c]X[c]X[c]}
        \arcsin(\sin x)=x,      & \arccos(\cos x)=x,               & \arctan(\tan x)=x,      \\
        \sin(\arcsin x)=x,      & \cos(\arccos x)=x,               & \tan(\arctan x)=x,      \\
        \arcsin(-x)=-\arcsin x, & \arccos(-x)=\symup\pi-\arccos x, & \arctan(-x)=-\arctan x, \\
                                & \cos(\arcsin x)=\sqrt{1-x^2}.
    \end{tblr}
\]
反三角函数最基本的作用,是在已知角的三角函数值时,拿来表示这个角(这在高中物理中偶尔会用到)。它身上也有许多奇妙
的性质,譬如,它们的导函数居然可以用有理式(或有理式的二次根式)表示。
\begin{example}
    求$(\arctan x)'$.
\end{example}
\begin{solve}
    先把这个函数写出来(这不是废话吗),替换其自变量,尝试把等号右边变得简单(换元是本题的关键)
    \[
        f(\tan x)=x
        .\]

    对两边求导,根据复合函数求导的法则\footnote{这里用到了$(\tan x)'=(\sin x / \cos x)'= 1 / \cos ^2x =1+\tan ^2 x$},有
    \[
        \bigl[\,f(\tan x)\bigr]'=f'(\tan x)({1+\tan ^2 x})=x'=1
        ,\]

    所以
    \[
        f'(\tan x)=\frac{1}{{1+\tan ^2 x}}
        .\]

    把自变量还原为$x$,得到:
    \[
        f'(x)=\frac{1}{1+x^2}
        .\]

    这就是我们要求的结果。
\end{solve}

\indent 上述变换的过程看起来有些不可思议。但实际上它用到了数学中的一个通法——换元(代换)。利用换元法,我们可以化繁为简,化不可能为可能。
今后在求极限、积分的时候,也会反复运用到这个方法。换元法的基本思路可以总结为:\textbf{用简单的变量替换复杂的变量;尽量把复杂未知的式子化成简单
    的已知式子}(例如之后会求的$\lim\limits_{x \to \infty}\left (1+\frac{1}{x}\right )^x=\mathrm e$的各种变式。)

\subsection{双曲函数与反双曲函数}
我们定义双曲函数如下表:

\begin{center}
    \begin{tblr}{colspec = {c|c|c|c}, row{2,3,4} = {abovesep=5pt,belowsep=5pt}}
        \hline
        函数名   & 定义                                                                         & 定义域       & 值域           \\
        \hline
        双曲正弦 & $\sinh x=\dfrac{\mathrm{e}^x-\mathrm{e}^{-x}}{2}$                            & $\mathbb{R}$ & $\mathbb{R}$   \\
        双曲余弦 & $\cosh x=\dfrac{\mathrm{e}^x+\mathrm{e}^{-x}}{2}$                            & $\mathbb{R}$ & $[1,+\infty )$ \\
        双曲正切 & $\tanh x=\dfrac{\mathrm{e}^x-\mathrm{e}^{-x}}{\mathrm{e}^x+\mathrm{e}^{-x}}$ & $\mathbb{R}$ & $(-1,1)$       \\
        \hline
    \end{tblr}
\end{center}

对双曲正弦,它有着与正弦完全类似的和差角公式(双曲余弦的和差角则略有不同)。特别地

\begin{gather}
    \cosh ^2 x-\sinh ^2 x=1,\\
    \sinh 2x=2\sinh x \cosh x,\\
    \cosh 2x=\cosh ^2 x+\sinh ^2 x.
\end{gather}
这些式子的证明都很简单。

反双曲函数的定义与反三角函数类似,几何上,将对应双曲函数的图像绕直线$x=y$“对折”即可。

我们尝试求出反双曲正弦的表达式:由反函数的定义,我们有$x=\sinh y$,即
\[
    x=\frac{\mathrm{e}^y-\mathrm{e}^{-y}}{2}.
\]
令$t=\symrm{e} ^y$,上式改写为
\[
    t^2-2xt-1=0.
\]
解之得$t=x\pm \sqrt{x^2+1}$,考虑到 $t=\mathrm{e}^y>0$ ,负值应舍去。故 $t=x+ \sqrt{x^2+1}$ .


由于$y=\ln t$,于是$y=\ln\left( x+ \sqrt{x^2+1} \right) $.


剩余若干反双曲函数的表达式请自行推导。\autoref{反三角函数、双曲函数与反双曲函数的图像} 中,我们给出反三角函数、双曲函数与反双曲函数的图像,试从中总结它们的性质。%补图


\begin{figure}[ht]
    \centering
    \makebox[0pt][c]{
        \begin{tikzpicture}
            \pgfplotsset{width = \linewidth/2+2em,height = 7cm}
            \begin{axis}[
                    axis x line = middle,
                    axis y line = middle,
                    every inner x axis line/.append style={->},
                    every inner y axis line/.append style={->},
                    title= {\itshape 反三角函数},
                    legend pos = south east,
                    xlabel = $x$,
                    ylabel = $y$,
                    ymajorgrids = false,
                    xmajorgrids = false,
                    grid style = dashed,
                    samples = 800,
                    xmin = -3, xmax = 3,
                    ymin = -2
                ]
                \addplot+[
                    no marks,
                    very thick,macred!90!black
                ] table {Plot1.dat};

                \addplot+[
                    no marks,
                    very thick,macyellow
                ] table {Plot2.dat};

                \addplot+[
                    no marks,
                    very thick,macgreen!90!black
                ] table {Plot3.dat};

                \addplot[
                    domain = 0:10,dashed
                ]{pi/2};

                \node[coordinate, pin=60:{\footnotesize$(1,\symup\pi /2)$}] at (axis cs:1,pi/2){};
                \node[coordinate, pin=230:{\footnotesize$(-1,\symup\pi)$}] at (axis cs:-1,pi){};
                \addplot [only marks,mark=*]
                coordinates { (1,pi/2) (0,pi/2) (-1,pi)};
                \legend{$\arcsin x$,$\arccos x$,$\arctan x$}
            \end{axis}
        \end{tikzpicture}
        \begin{tikzpicture}
            \pgfplotsset{width = \linewidth/2+2em,height = 7cm}
            \begin{axis}[
                    axis x line = middle,
                    axis y line = middle,
                    every inner x axis line/.append style={->},
                    every inner y axis line/.append style={->},
                    title={\itshape 双曲函数},
                    legend pos = south east,
                    xlabel = $x$,
                    ylabel = $y$,
                    ymajorgrids = false,
                    xmajorgrids = false,
                    grid style = dashed,
                    samples = 1000,
                    xmin = -3, xmax = 3,
                    ymin = -2, ymax = 5
                ]
                \addplot+[
                    no marks,
                    very thick,macred!90!black
                ]{sinh(x)};

                \addplot+[
                    no marks,
                    very thick,macyellow
                ]{cosh(x)};

                \addplot+[
                    no marks,
                    very thick,macgreen!90!black
                ]{tanh(x)};

                \addplot[
                    domain = 0:10,dashed
                ]{1};

                \addplot [only marks,mark=*]
                coordinates { (0,1) };

                \node[coordinate, pin=230:{\footnotesize$(0,1)$}] at (axis cs:0,1){};
                \legend{$\sinh x$,$\cosh x$,$\tanh x$}
            \end{axis}
        \end{tikzpicture}
    }

    \begin{tikzpicture}
        \pgfplotsset{width = \linewidth ,height = 8cm}
        \begin{axis}[
                axis x line = middle,
                axis y line = middle,
                every inner x axis line/.append style={->},
                every inner y axis line/.append style={->},
                title={\itshape 反双曲函数},
                legend pos = south east,
                xlabel = $x$,
                ylabel = $y$,
                ymajorgrids = false,
                xmajorgrids = false,
                grid style = dashed,
                samples = 1000,
            ]
            \addplot+[
                no marks,
                very thick,macred!90!black
            ] table {Plot4.dat};

            \addplot+[
                no marks,
                very thick,macyellow
            ] table {Plot5.dat};

            \addplot+[
                no marks,
                very thick,macgreen!90!black
            ] table {Plot6.dat};


            \legend{$\operatorname{arcsinh}x$,$\operatorname{arccosh}x$,$\operatorname{arctanh}x$}
        \end{axis}
    \end{tikzpicture}
    \caption{反三角函数、双曲函数与反双曲函数的图像}
    \label{反三角函数、双曲函数与反双曲函数的图像}
\end{figure}


\section{“充要条件”的理解}
在部编版高中数学教材的必修一(旧版的选修2-1)中,大家已经学习了简易的逻辑术语,包括特称、全称量词,
以及充分条件、必要条件与充分必要条件的定义。但实际上,大家只有在选择填空题中才会见到它们的身影(特别是几种“条件”)%
而很少在证明题中应用到它们。下面我们就通过一些例子来更加深入的理解这几种“条件”。
\subsection{必要条件——自信的估计}
我们知道$p$是$q$的必要条件可以用$q \Rightarrow {p}$来表示,但仅有符号无法帮助我们很好地理解抽象的概念。
举一个简单的例子:\textit{“幽幽子吃东西”}是\textit{“幽幽子吃饱饭”}的必要条件。因为\textbf{不(表示否定) }\textit{“吃东西”}就一定%
\textbf{不可能(同样表示否定)}\textit{“吃饱饭”};由\textit{“幽幽子吃饱饭”}这一事实,我们就一定可以得出\textit{“幽幽子吃东西”}这一前提
条件。(即$q$:\textit{“幽幽子吃饱饭”} $\Rightarrow p$:\textit{“幽幽子吃东西”})


从更“数学”一点的角度看,我们在高中会遇到这样一类导数题——已知函数满足一定的不等条件,求参数的取值范围。
这时,“必要性探路\footnote{先取某点的函数值解出参数的一个取值范围,再证明这个取值范围是恒成立的}”往往是很常用
的方法,但它有时候也会失效。
\begin{example}
    当$x \geqslant 0$ 时,$\mathrm{e}^x+ax^2 -x \geqslant \frac{1}{2}x^3 +1$,求$a$的取值范围。
\end{example}
\begin{solve}
    本题是2020年高考全国一卷理科数学的导数大题,很明显,本题可以用分离参数的方法解出。
    但如果我们用必要性探路,会有什么后果呢?

    记
    \[
        f(x)=\mathrm{e}^x+ax^2 -x-\left (\frac{x^3}{2} +1\right )
        .\]
    容易发现$f(0)=0$,那么必须有$f'(0)\geqslant 0$,
    而$f'(x)=\mathrm e^x-\frac{3}{2}x^2+2ax-1.$,则
    $f'(0)=0$ 那么又必须有$f''(0)\geqslant 0$,而$f''(x)=\mathrm e^x-3x+2a$,则$a\geqslant -\frac{1}{2}$.

\end{solve}

然而,这并非正确的答案。事实上,利用分离参数法,我们解得$a$的取值范围为$\left [ \frac{7-\mathrm e^2}{4},+\infty  \right ) $,
在$x=2$时取得最小值。这说明“便捷”的必要性探路并非万能。究其原因,必要条件是“被扩大的前提”。在它之中,仅有一部分
能够推断出“结果”$q$,这就好比说“吃了东西的幽幽子不一定能吃饱”。

\subsection{充分条件——不一定完备的前提}
比起必要条件,充分条件理解起来似乎轻松一些。正如$p\Rightarrow q$中的右箭头一样,它的定义符合我们一般的思维
顺序——由因及果。需要注意的是,一个结果可以对应多种原因,因此,充分条件是不唯一的。

\subsection{充要条件——终极目标}
可以这么说:充要条件是数学中最精美的需要。一个命题被提出后,只有找到它的充要条件,才能说它得到了解决。


拆解充要条件的符号``$\Leftrightarrow $",我们发现它由``$\Leftarrow$" ``$\Rightarrow $"两部分组成。
这看起来像是在说废话,实际上却蕴含了这么一种思想:如果你想证明两个命题等价,只需要证明由任意一方可以推出
另外一方即可。还是像废话?我们来看一个实际的思路:如果$a,b$之间满足一定关系,证明$a=b$不仅可以由等量关系推出,也可
分别证明$a\geqslant b$,~$b\geqslant a$,从而得出$a=b$.
\begin{example}
    已知集合$A=\bigl\{\,x \bigm| x=2m-1,\,m\in \mathbb{Z} \,\bigr\}$,~$B=\bigl\{\,x \bigm| x=4k\pm 1,\,k\in \mathbb{Z} \,\bigr\} $,求证:$A=B$.
\end{example}
\begin{glue}
    这里我们需要证明两个集合相等.
    
    如果读者对$ZF$公理体系有所了解的话,在$ZF$公理体系中,外延公理(Axiom of Extensionality)说明了集合相等的条件:
    
    \[
        \forall x \forall y \[\forall z (z\in x \Leftrightarrow z \in y) \Rightarrow x = y\].
    \]

    注意这里的前提为$z\in x \Leftrightarrow z \in y$,
    因此我们需要证明$z\in x \Leftarrow z\in y$和$z\in x \Rightarrow z\in y$同时成立.

    一个很常见的且(一点也不有趣的)做法是采用集合包含的定义:
    \[
        S \subseteq T \Leftrightarrow \forall x : (x\in S \Leftarrow x\in T)
    \]
    
    很显然,$x = y \Leftrightarrow x \subseteq y \land y \subseteq x$.
\end{glue}
\begin{prove}
    ($A\subseteq B$)一方面,若$x\in A$,则当$m=2k$时,~$x=4k-1$且$k\in \mathbb{Z}$,


    当$m=2k+1$时,$x=2(2k+1)-1=4k+1$,~$k\in \mathbb{Z}$,从而$x\in B$,即得$A\subseteq B$.


    \newline
    ($B \subseteq A$)另一方面,若$x∈B$,则$x=4k±1$,~$k\in \mathbb{Z}$.


    当$x=4k-1$时,$x=2(2k)-1$,令$m=2k\in \mathbb{Z}$,有$x=2m-1\in A$.


    当$x=4k+1=2(2k+1)-1$,令$m=2k+1\in \mathbb{Z}$,有$x=2m-1\in A$.


    从而$x\in A$,即得$B\subseteq A$,综合以上两方面即得$A=B$.
\end{prove}


我们再用一个例子简单地介绍证明“$p$是$q$的充要条件”的思路。
\begin{example}
    已知$a$,~$m$,~$n\in \mathbb{N^*}$,求证:$a^m-1\mid a^n-1$的充要条件是$m\mid n$.
\end{example}
\begin{prove}
    充分性:由$m\mid n$可设$n=km$,~$(k\in \mathbb{Z})$,我们有
    \[
        \begin{aligned}
            a^m-1=(a-1)(1+a+a^2+\dots +a^{m-1}), \\
            a^n-1=(a-1)(1+a+a^2+\dots +a^{km-1}).
        \end{aligned}
    \]
    上式可由等比数列求和公式得到。又
    \[
        \begin{aligned}
             & \mathrel{\phantom{=}} (a-1)\left( a+a^2+\dots +a^{km-1} \right)               \\
             & =  1+a+a^2+\dots +a^{m-1}+a^m+\dots+a^{2m}+\dots+a^{km-1}                     \\
             & =  (a-1)\left( 1+a+a^2+\dots +a^{m-1})(1+a^m+a^{2m}+\dots +a^{(k-1)m} \right) \\
             & =  (a^m-1)\left( 1+a^m+a^{2m}+\dots +a^{(k-1)m} \right) .
        \end{aligned}
    \]
    故$a^m-1\mid a^n-1$.充分性得证


    必要性:设 $a^m=t$,则$a^n=(a^m)^{\frac{n}{m}}=t^{\frac{n}{m}}$.由
    \[
        t^{\frac{n}{m}}-1=(t-1)\left( 1+t+t^2+\dots +t^{\frac{n}{m}-1} \right)
        .\]
    是整数且$t-1\mid t^{\frac{n}{m}}-1$可知
    $\frac{n}{m}$是整数,由此$m\mid n$得证.


    综上所述,$a^m-1\mid a^n-1$的充要条件是$m\mid n$.

\end{prove}

上面这个例子向我们展示了证明充要性的基本方式:分别证明充分性和必要性。很多情况下,某一方面的证明需要利用
反证法,而且两个方面的证明思路互相提示。

下面这个著名的例子使用了反证法.

\begin{example}
    质数有无穷多个
\end{example}
\begin{proof}
    假设质数个数有限,设其为$p_1, p_2, p_3, \ldots, p_n$, $p_i$为第$i$个质数. $n\in \mathbb{N}$
    例如$p_1 = 2$, $p_2 = 3$.

    构造$p = \Pi_{i=1}{n}p_i + 1$, 那么$p_i \not\mid p, i=1,2,3,\ldots,n$.

    因此$p$是质数,因为它不是合数,否则它可以被某个小于$p$的质数整除.

    因此,质数有无穷多个.
\end{proof}
\begin{questionbox}
    这个证明是怎么说明质数有无穷多个的?
\end{questionbox}

\section{常用方法——放缩、夹逼与归纳}
\subsection{放缩法}
进入高等数学的学习之后,我们不会再像高中那样特意地证明一些不等式。但在证明某些命题,或者求极限的时候,仍需要
证明不等式,这个时候放缩法的使用就显得尤为重要。这一节,我们会针对高等数学(数学分析)的学习需要,介绍一些常
用的放缩技巧和思路。
\subsubsection{放缩的常用工具}


\begin{itemize}
    \item \textbf{与绝对值有关的不等式}

          关于绝对值,我们熟知有以下不等式:
          \begin{equation}
              \big | \lvert a \rvert-\lvert b\rvert  \big |
              \leqslant \lvert a\pm b\rvert \leqslant \lvert a\rvert +\lvert b\rvert
              .
              \label{三角不等式}
            \end{equation}

          $a$,~$b$间取加号时,左边等号的成立条件是$a=-b$,右边等号的成立条件是$a=b$同号。取减号是恰好相反。

          在之后证明收敛数列极限唯一、极限乘法规则和柯西审敛准则时,都会用到这一工具。
          \begin{example}
              求证:若$\lim\limits_{n \to \infty} a_n =a$,则$\lim\limits_{n \to \infty} |a_n| =|a|$.
          \end{example}
          \begin{prove}
              由已知,对于任意正数$\varepsilon$,总存在正整数$N$,使得当$n>N$时,总有$|a_n-a|<\varepsilon$.


              由绝对值不等式,当$n>N$时,总有$\big ||a_n|-|a|\big |\leqslant |a_n-a|<\varepsilon$.
              因此$\lim\limits_{n \to \infty} |a_n| =|a|$.
          \end{prove}
          另外,对于任意$n$个数,上述不等式还有拓展:
          \[
              |\;\!a_1+a_2+\cdots +a_n|\leqslant |a_1|+|a_2|+\cdots +|a_n|
              .\]

          取等条件为$a_1,\,\dots\,,a_n$全部同号。
    \item \textbf{与三角函数有关的不等式}

          我们可以通过几何方法证明:
          \[
              \sin x<x<\tan x,\quad\left( 0<x<\frac{\symup\pi}{2} \right)
              .\]

          具体过程可以参照%加文献引用
          各类高数或数分教材。感兴趣的读者也可以自行思考证明方式.
          然而,在刚接触高数时,我们往往不需要用到这么精细的放缩。注意到正弦和余弦函数的绝对值均不大于 $1$,利用
          这一性质就可以解决很多问题了。

    \item \textbf{与指对数有关的不等式}

          我们知道$\lim\limits_{n \to \infty}\left (1+\frac{1}{n}\right )^n=\mathrm e$.事实上,
          数列$\left\{ \left (1+\frac{1}{n}\right )^n  \right\} $是递增的,证明过程如下:
          \begin{prove}
              由均值不等式,
              \begin{equation}
                  \left (1+\frac{1}{n}\right )^n=1\cdot \left (\frac{n+1}{n}\right )\left (\frac{n+1}{n}\right )
                  \cdots \left (\frac{n+1}{n}\right )<
                  \left (\frac{1+n\cdot \frac{n+1}{n}}{n}\right )^{n+1}=\left (1+\frac{1}{n+1}\right )^{n+1}\label{与指对数有关的不等式}
              \end{equation}
          \end{prove}


          另外,我们还可以证明 $\lim\limits_{n \to \infty}\left (1+\frac{1}{n}\right )^{n+1}=\mathrm e$,且数列
          $\left\{  \left (1+\frac{1}{n}\right )^{n+1} \right\}  $是递减的。结合以上事实,我们得到:
          \[
              \left (1+\frac{1}{n}\right )^n<\mathrm e<\left (1+\frac{1}{n}\right )^{n+1\!\!\!\!\!\!\!\!\!\!\!\!}
              ,\]


          取对数后,我们得到一个很有用的结论:
          \[
              \frac{1}{n+1}<\ln \left( 1+\frac{1}{n} \right) <\frac{1}{n}, \quad (n\in \mathbb{N} ).
          \]


          我们在高中时利用导数工具得到过这个结论,事实上,在正整数范围内,它可以仅由数列极限的知识得到。

    \item \textbf{与阶乘有关的不等式}

          阶乘有许多重要的性质,这里我们只介绍在不等关系方面的性质。

          可以这么说,阶乘是比常数的指数更“大”的存在,见     下面的例子。
          \begin{example}
              求证:$\lim\limits_{n \to \infty} \frac{n!}{2^n} =0$.
          \end{example}
          \begin{prove}
              \[
                  \frac{n!}{2^n}=\frac{2\times 2\times \cdots\times 2}{1\times 2\times \cdots\times n}
                  <2\cdot \frac{2}{n}=\frac{4}{n}
                  .\]

              根据阶极限定理,结合常用极限$\lim_{n\to \infty}\frac{1}{n} = 0$易得结论,
              此处略去具体过程.
          \end{prove}


          \begin{example}
              求证:对$n\in \mathbb{N}$,$n!>\left (\frac{n}{\mathrm e} \right )^n$.\label{例子9}
          \end{example}
          \begin{prove}
              我们取数列$a_n=\left (\frac{n}{\mathrm e} \right )^n$,则
              \begin{gather}
                  \frac{a_n}{a_{n-1}}=\frac{n^n}{(n-1)^{n-1}\mathrm e}=\frac{n\left (1+\frac{1}{n-1}\right )^{n-1}}{\mathrm e},\\
                  a_n=\frac{a_n}{a_{n-1}} \frac{a_{n-1}}{a_{n-2}}\cdots \frac{a_2}{a_1} a_1< n(n-1)\cdots 2a_1<n(n-1)\cdot \cdots \cdot 2\cdot 1=n!.
              \end{gather}
              证毕。
          \end{prove}

          例 \autoref{例子9} 揭示了$n$的阶乘与$n$的$n$次幂之间的关系,它与重要极限$\lim\limits_{n \to \infty}\left (1
              +\frac{1}{n}\right )^n=\mathrm e$密切相关。事实上,我们有极限
          $\lim\limits_{n \to \infty}\frac{n}{\sqrt[n]{n!}}=\mathrm e$.
\end{itemize}

\subsubsection{放缩的常用手段}
\label{sssec:A}
许多时候,面对一些形式复杂的式子,我们一时想不到如何对其进行放缩。下面我们将用几个例子介绍放缩的常用手段。

\paragraph{朝着可以化简的方向放缩}

要证明某些累加式或累乘式的值在某个范围内,我们一般把舍弃某些项,把它们放缩成可以求和或求积的形式。
\begin{example}
    求证:$\left ( 1+\frac{1}{2}\right )\left ( 1+\frac{1}{2^2}\right )
        \cdots \left ( 1+\frac{1}{2^n}\right )<\mathrm{e}$.
\end{example}
\begin{prove}
    由 \eqref{与指对数有关的不等式} 的最后一个结论,我们有$\ln \left ( 1+\frac{1}{2^k}\right )<\frac{1}{2^k}(k=1,2,\cdots,n)$.则
    \[
        \ln\left( \left ( 1+\frac{1}{2}\right )\left ( 1+\frac{1}{2^2}\right )
        \cdots \left ( 1+\frac{1}{2^n}\right ) \right) <\frac{1}{2}+\frac{1}{2^2}
        +\cdots+\frac{1}{2^n}=\frac{1}{2}\cdot \frac{1-\frac{1}{2^n}}{1-\frac{1}{2}}<1
        .\]
    从而$\left ( 1+\frac{1}{2}\right )\left ( 1+\frac{1}{2^2}\right )
        \cdots \left ( 1+\frac{1}{2^n}\right )<\mathrm{e}$.
\end{prove}


上面的例子中,我们观察到待证式左边暗含等比数列,就想办法将其提取出来。
恰好,取对数之后放缩可以把“$1$”消去,便完成了化简的工作。

\paragraph{待定系数,先猜后证}

某些不等式,特别是证明某式子大于或小于某个非0常数,(这在求极限时很常见)可以用待定系数的方法来进行放缩。
\begin{example}
    求证:$\lim\limits_{n \to \infty}\sqrt[n]{n}=1.$
\end{example}
\begin{prove}
    下面我们将会用到3.2节会学到的夹逼定理——事实上,我们的目标是证明$\sqrt[n]{n}-1$小于
    一个极限为$0$的数列。观察到$\sqrt[n]{n}-1$的特点,我们发现移去$1$对原式取$n$次方后可以实现
    有效的化简。


    令$\sqrt[n]{n}=1+\lambda_n$,则
    \[
        n=(1+\lambda_n)^n=1+n\lambda_n+\frac{n(n-1)}{2}\lambda^2 _n+\cdots >1+\frac{n(n-1)}{2}\lambda^2 _n
        .\]
    由此解得$\lambda_n<\sqrt{\frac{2}{n}}$,因此$0<\sqrt[n]{n}-1<\sqrt{\frac{2}{n}}$.由夹逼
    定理可知$\lim\limits_{n \to \infty}\sqrt[n]{n}=1.$
\end{prove}


关于这种技巧,在初等数学的范围内也有很多应用,例如用均值不等式、柯西不等式等解题时,利用取等条件限制,
解出配凑的系数。类似的例子还有很多,限于篇幅,就不在此处赘述了。


总之,放缩法的技巧和工具五花八门。只有勤加练习,多积累有关知识,才能得心应手地应用。然而,我们
也没有必要像高中时那样,为了应试,而刻意去寻找各种偏、怪的不等式,掌握常用且有用的那部分即可。
% \end{enumerate}



\subsection{夹逼定理}
在几乎所有高数教材中,我们都会看到这样一条定理
\begin[夹逼定理(Sandwitch Theorem)]{theorem}
    若数列$\{ b_n \} $和$\{ c_n \} $都收敛于$a$,且对
    所有充分大的$n$,有$b_n \leqslant a_n \leqslant c_n$,则数列$\{ a_n \} $也收敛,且极限也为$a$.
\end{theorem}


关于夹逼定理的证明,请参照各种教材。我们之所以在此特别提到夹逼定理,是因为其背后蕴含着一个重要思想——夹逼思想,即
“从两边往中间靠”。事实上,这一思想贯穿我们的数学学习历程。
\begin{example}
    已知函数$f(x)=ax^2+bx+c$的图像过点$(-1,0)$,且对$x\in \mathbb{R}$都有$4x-12\leqslant f(x)\leqslant 2x^2-8x+6$.求$f(x)$.
\end{example}
\begin{solve}
    这道题节选自$2021$年广东省中考数学卷$25$题,其关键一步就利用了夹逼思想。


    当$x=3$时,$4x-12=2x^2-8x+6=0$,则$f(3)=0$,即$f(x)=a(x+1)(x-3)=ax^2-2ax-3a$.


    又$4x-12\leqslant f(x)$,则$ax^2-(2a+4)x+12-3a\geqslant 0$,即
    \[
        \Delta = (2a+4)^2-4a(12-3a)=16a^2-32a+16=0
        .\]
    解得$a=1$.


    从而 $f(x)=x^2-2x-3$.
\end{solve}
无论是初等的不等式问题,还是高等的求极限,夹逼思想都非常重要。其思路也类似于放缩的基本手段——用可直接求极限的
式子,去“夹”出不那么容易直接求出极限的式子。总而言之,就是由未知联想已知,再由已知导出未知。下面的例子就是
夹逼定理在求极限中的经典应用。
\begin{example}
    求$\lim\limits_{n \to \infty}\left( \frac{1}{\sqrt{n^2+1}}+\frac{1}{\sqrt{n^2+2}}+\cdots +\frac{1}{\sqrt{n^2+n}} \right) .$
\end{example}
\begin{solve}
    注意到
    \[
        \frac{n}{\sqrt{n^2+n}}\leqslant \frac{1}{\sqrt{n^2+1}}+\frac{1}{\sqrt{n^2+2}}+\cdots +\frac{1}{\sqrt{n^2+n}}
        \leqslant \frac{n}{\sqrt{n^2+1}}
        .\]
    而
    \[
        \lim\limits_{n \to \infty}\frac{n}{\sqrt{n^2+n}}=\lim\limits_{n \to \infty}\frac{1}{\sqrt{1+\frac{1}{n}}}=1,\quad
        \lim\limits_{n \to \infty}\frac{n}{\sqrt{n^2+1}}=\lim\limits_{n \to \infty}\frac{1}{\sqrt{1+\frac{1}{n^2}}}=1
        .\]
    由夹逼定理即得$\lim\limits_{n \to \infty}\left( \frac{1}{\sqrt{n^2+1}}+\frac{1}{\sqrt{n^2+2}}+\cdots +\frac{1}{\sqrt{n^2+n}} \right) =1$.
\end{solve}
% \def\thesubsubsection{\arabic{subsubsection}}
上面的例子也用到了 \autoref{sssec:A} 提到的放缩手段——朝着可以化简的方向放缩。
\subsection{数学归纳法}
\begin{quote}
    \emph{Mathematical Induction} is a method for \href{https://en.wikipedia.org/wiki/Mathematical_proof}{proving}
    that a statement $P(n)$ is true for every \href{https://en.wikipedia.org/wiki/Natural_number}{natural number} $n$,
    that is, that the infinitely many cases $P(0), P(1), P(2), P(3), \ldots$ all hold.
    
    \begin{flushright}
        --- Wikipedia, \url{https://en.wikipedia.org/wiki/Mathematical_induction}
    \end{flushright}
\end{quote}

在高中时,我们已经对数学归纳法有所了解,但并没有深入地研究其应用,并且仅局限于常见的第一数学归纳法。下面我们更深入地介绍一下
数学归纳法的奇妙之用。
\subsubsection{第一数学归纳法}
第一数学归纳法,顾名思义,就是我们最常用的归纳法。其具体内容就不在此处展开,我们通过一个例子来帮助大家回忆一下它的使用。
\begin{example}
    设$a_n=\sqrt{2+\sqrt{2+\sqrt{2+\cdots}}}$($n$重根式),求证$\lim\limits_{n \to \infty}a_n$存在并求其极限。
\end{example}
\begin{solve}
    $a_n$满足递推关系:$a_1=\sqrt{2}$, $a_{n+1}=\sqrt{a_n+2}$.

    注意到$a_2=\sqrt{2+\sqrt{2}}>a_1$, $a_3=\sqrt{2+\sqrt{2+\sqrt{2}}}>a_2$.

    我们猜想,如果$a_n>a_{n-1}$,那么
    \[
        a_{n+1}-a_n=\sqrt{a_n+2}-\sqrt{a_{n-1}+2}=
        \frac{a_n-a_{n-1}}{\sqrt{a_n+2}+\sqrt{a_{n-1}+2}}>0
        .\]
    从而$a_{n+1}>a_n$对所有自然数$n$成立,由数学归纳法可得$\{ a_n\}$单调递增。


    同样由数学归纳法可以证得$a_n<2$,因此$\{ a_n\}$单调递增有上界,则$\{ a_n\}$收敛。设其极限为$a$,在递推式两边取极限
    可以解得$a=2$(负值舍去),从而$\lim\limits_{n \to \infty}a_n=2$.
\end{solve}
这个例子中,我们两次运用数学归纳法证明了很“显然”,却不方便用常规方法证明的结论,可见其作用强大。
\subsubsection{第二数学归纳法}
第二数学归纳法与第一数学归纳法类似,但却比它更强,其内容如下:

\begin{enumerate}
    \item 前提:当$n=m$,~$(m \in \mathbb{N})$时,结论成立;
    \item 假设与归纳:假设$n\leqslant k$(注意与第一数学归纳法比较)时结论成立,若由此推得$n=k+1$时结论也成立,
          则结论对$n \geqslant m$总成立。

\end{enumerate}

我们可以利用高中数学教材中多米诺骨牌的例子来理解:第一数学归纳法是“前一块骨牌倒下”推出“后一块骨牌倒下”,
而第二数学归纳法是“前面所有的骨牌倒下”推出“后一块骨牌倒下”,它同样是合理的。


\begin{example}
    【\textup{Chebyshev}(切比雪夫)多项式】求证:$\cos n\theta$,~$(n \in \mathbb{N})$可以表示为关于$\cos \theta$的整系数多项式。
\end{example}
\begin{prove}
    首先理解题干意思:我们要将$\cos n\theta$的
    展开式写成$a_0+a_1\cos \theta+a_2\cos^2 \theta+\cdots +a_n \cos^n\theta$的形式,其中$a_k$均为整数。


    我们考虑积化和差公式(见 \autoref{积化和差} ),有$2\cos k\theta\cos \theta=\cos (k+1)\theta +\cos (k-1)\theta$.
    显然$n=1$,~$2$时$\cos n\theta$,~$(n \in \mathbb{N})$均可以表示为关于$\cos \theta$的整系数多项式\footnote{$n=2$时即二倍角公式。}。
    假设$n\leqslant k$时$\cos n\theta$均可以表示为关于$\cos \theta$的整系数多项式,则
    \[
        \cos (k+1)\theta=2\cos k\theta\cos \theta-\cos (k-1)\theta
        .\]
    是整系数多项式的减法运算,故$\cos (k+1)\theta$仍为整系数多项式,由第二数学归纳法知结论成立。
\end{prove}
仔细体会上面的证明过程,你会渐渐认识到数学归纳法的强大威力。

\section{参数方程与坐标系补充}
\subsection{参数方程简介}
平面直角坐标系上的参数方程,就是分别用参量$t$定义坐标分量$x,y$,$x,y$之间通过$t$形成某种关系
(很多时候可以统一到一个方程里),并在坐标系上体现为曲线。简而言之,参数方程是曲线方程(函数图像)的一种形式。
\begin{example}
    $ x=r\cos \theta$,~$y=r\sin \theta$是圆的参数方程,可以化为圆的标准方程$x^2+y^2=r^2.$
\end{example}
\begin{example}
    $ x=a\cos \theta$,~$y=b\sin \theta$ 是椭圆的参数方程,可以化为椭圆的标准方程
    $\displaystyle \frac{x^2}{a^2}+\frac{y^2}{b^2}=1.$
\end{example}
\begin{example}\label{li18}
    滚轮线是匀速直线运动和匀速圆周运动的叠加运动的轨迹。一个半径为$r$刚体圆形滚轮沿直线往前滚动$\theta$角时,质心前进的距离是 $r\theta $,
    高度保持为$r$ 不变,其初始坐标记作 $(0,r)$. 设$\theta =0$时轮沿上某点$P$接触地面,其坐标为$(0,0)$. 则滚动 $\theta$ 角后,
    $P$ 点的运动 即质心匀速直线运动与$P$ 点相对质心匀速圆周运动的线性叠加,满足:
    $(x,y) = (r\theta ,r) + (-r\sin \theta,-r\cos \theta) = \bigl(r(θ - \sin θ),r(1 - \cos θ)\bigr)$.这样就得出一个$x,y$关于
    参数$\theta $的参数方程,称为滚轮线或摆线。

    \begin{center}
        \begin{tikzpicture}
            \pgfplotsset{width=14cm,height=7cm}
            \begin{axis}[
                    title={当 $r=1$ 时的摆线},
                    xmin = -2, xmax = 7 ,
                    ymin = -2, ymax = 4.5,
                    axis x line=middle,
                    axis y line=middle,
                    every inner x axis line/.append style={->},
                    every inner y axis line/.append style={->},
                    xlabel=$x$,
                    ylabel=$y$,
                ]
                \addplot [
                    variable = t,
                    domain = 0 : 2*pi,
                    very thick,
                    trig format plots = rad,
                    samples = 1000,
                ] (
                {t - sin(t)},
                {1 - cos(t)}
                );
            \end{axis}
        \end{tikzpicture}
    \end{center}
\end{example}
事实上,并非所有参数方程都可以简单地消去参数,转化为关于$x$,~$y$的方程。(如例 \autoref{li18} 中的滚轮线)但它们仍可以进行求切线斜率等操作。


\subsection{极坐标、柱坐标与球坐标}
\subsubsection{极坐标系}
极坐标系由一个原点——极点,和以极点为端点的一条射线——极轴构成。
与我们熟知的平面直角坐标系类似,极坐标系也包含两个坐标分量:与极点的距离——极径$\rho $,
以及与极轴正方向的夹角——极角$\theta $,
其中$\rho $都只能取非负值,而$\theta $可以取任意实数值,且每增加$2\symup\pi$,就相当于绕极点“转了一圈”。


容易发现,极坐标与直角坐标有转换关系如下:
\[
    \rho ^2=x^2+y^2,\quad\rho \cos \theta=x,\quad\rho \sin \theta=y
    .\]


对于一些更“对称”的曲线来说,用极坐标描述它们,比用直角坐标系简单便捷得多。
\begin{example}
    以圆锥曲线的一个焦点(椭圆取左焦点,双曲线取右焦点)为极点,以过焦点的对称轴为极轴(向右为正方向)
    ,其极坐标方程可以表示为
    \[
        \rho =\frac{ep}{1-e\cos \theta},\quad(0\leqslant \theta < 2 \symup\pi)
        .\]
    其中$e$为圆锥曲线的离心率,$p$为焦准距。
\end{example}


这一极坐标方程暗含圆锥曲线的统一定义,足见极坐标系的优势所在,其推导过程请读者自行尝试。%....
\begin{example}
    心形线$\rho =1-\cos \theta$,~$ (0\leqslant \theta < 2 \symup\pi)$若改写为直角方程,其表达式将变为
    $(x^2+y^2+x)^2=x^2+y^2$,不够直观。
\end{example}

\begin{center}
    \begin{tikzpicture}
        \pgfplotsset{width=7cm,height=7cm}
        \begin{polaraxis}
            \addplot [
            samples = 400,
            domain = 0:2*pi,
            very thick,
            trig format plots = rad,
            ] (deg(x), {1 - cos(x)});
        \end{polaraxis}
    \end{tikzpicture}
\end{center}

以上两个例子向我们展示了极坐标系适用的几个场景。此外,阿基米德螺线,玫瑰线等
都是极坐标的典型例子。

\begin{center}
    \begin{tikzpicture}
        \pgfplotsset{width=7cm,height=7cm}
        \begin{polaraxis}[title = {\itshape 阿基米德螺线 $\rho =4+\theta$,~$(0\leqslant \theta\leqslant 4\symup \pi )$}]
            \addplot [
            samples = 400,
            domain = 0:4*pi,
            very thick,
            trig format plots = rad,
            ] (deg(x), {4 + x});
        \end{polaraxis}
    \end{tikzpicture}
    \quad
    \begin{tikzpicture}
        \pgfplotsset{width=7cm,height=7cm}
        \begin{polaraxis}[title = {\itshape 玫瑰线 $\rho =5\sin (4\theta)$,~$(0\leqslant \theta\leqslant 2\pi )$}]
            \addplot [
            samples = 400,
            domain = 0:2*pi,
            very thick,
            trig format plots = rad,
            ] (deg(x), {5*sin(4*x)});
        \end{polaraxis}
    \end{tikzpicture}
\end{center}

\subsubsection{柱坐标与球坐标简介}
柱坐标可以看作在极坐标系平面上,再“拉”出一条垂直的$z$轴。我们熟知的等距螺线就可以方便地用它来表示。

球坐标系是极坐标系在空间中的延伸。它有两个角分量:连线与正$z$轴(垂直轴)的夹角——天顶角$\theta$,以及
连线在$xy$平面的投影与正$x$轴的夹角——方位角$\phi$.它在研究球对称的情况时十分便捷。
\section{极限思维与实数理论概述}
高中时有位数学老师的话让我印象深刻:“没学过微积分,人半辈子都是黑暗的哦。”这种看法虽然比较极端,但也暗示
了微积分对人思维的重要性。其中,最重要的一点,就是如何用精确的数学语言去描述看似浅显的“极限”“连续”概念。

\subsection{芝诺悖论和极限的提出}

\subsection{从定义看极限}\label{continous property}
高数课本上重点阐述了关于数列极限的“$\varepsilon  \text{-} N$语言”和关于函数极限的“$\varepsilon  \text{-} \delta$语言”,
这里我们重点从数列极限的角度切入,帮助大家弄懂极限是什么,要怎么证明有关收敛性的题目。


仔细观察$\varepsilon  \text{-}N$语言的内容:若对于所有正数$\varepsilon$,均存在正整数$N$,使得对于所有大于$N$
的整数$n$,都有$|\;\!a_n-a|<\varepsilon$.记住$\varepsilon$是可以\textbf{任意}小的,只要它比0大,取
什么值都没问题。但我们的$|a_n-a|$居然比它还要小!也就是说,只要有无穷多个(可以说明,这等价于“存在正整数$N$,使得对于所有大于$N$
的整数$n$”)$|\,a_n-a\,|$比任意小的$\varepsilon$还要小,就可以说$a_n$收敛于$a$.


这样,我们就用自然语言描述了极限的意思——无论多小,总还存在更小的。需要注意的是,这种描述有道理,但并不完全
准确。而下面我们就来介绍如何准确地理解和应用极限的定义。
\begin{example}
    求证:若$a_n>0$,且 $\lim\limits_{n \to \infty}\frac{a_n}{a_{n+1}}=l>1$,则 $\lim\limits_{n \to \infty}a_n=0$.
\end{example}
\begin{prove}
    题设等价于$\lim\limits_{n \to \infty}\frac{a_{n+1}}{a_n}=\frac{1}{l}=t<1$,
    即
    \[
        \forall \varepsilon>0,\,\exists N\in \mathbb{N},\,\forall n>N,\,\left|\, \frac{a_{n+1}}{a_n}-t\, \right| <\varepsilon
        \Rightarrow \frac{a_{n+1}}{a_n}<t+\varepsilon
        .\]
    我们取$\varepsilon=1-t-\tau $,其中$\tau >0$且$\tau +t<1$.即得$\frac{a_{n+1}}{a_n}<1-\tau $.
    因此,$\forall n>N,$
    \[
        a_n=a_{N+1}\frac{a_{N+2}}{a_{N+1}}\cdots \frac{a_n}{a_{n-1}}<a_{N+1}(1-\tau)^{n-N-1}
        .\]
    对于任意正数$\delta$,我们取$n>\left\lfloor\log_{1-\tau}\frac{\delta}{\scriptstyle{a}_{{N+1}}}  \right\rfloor+N+2$,则有
    \[
        a_n<a_{N+1}(1-\tau)^{\left\lfloor \log_{1-\tau}(\delta / a_{N+1}) \right\rfloor+1}
        <a_{N+1}(1-\tau)^{\log_{1-\tau}(\delta / a_{N+1})}=\delta
        .\]
    又$a_n>0$,从而$\lim\limits_{n \to \infty}a_n=0$.
\end{prove}
上面的例子中,我们直观地感觉到$a_n$类似于“公比小于1的等比数列”,再利用极限的定义,构造放缩得到“等比数列”。
从始至终,极限的定义都得到了应用。


无论如何,最基本的定义或定理都是许多题目的关键,熟练掌握的重要性不言而喻。

\subsection{实数完备性的理解}
我们知道,数的概念发展经历了由自然数到整数,再到有理数,最后到实数和复数的过程。其中,有理数与实数
的辨析是这一节内容的重点。


我们知道,一个数是有理数的充要条件是,它可以被表示为$\frac{q}{p}$的形式,其中$p$,~$q$是互质(最大公因数为1)
的整数。我们说有理数是稠密的,是指任意两个实数之间必存在有理数。(利用前述的放缩法可以证明,这作为一道小小
的思考题)


但它并不是连续(参照 \autoref{continous property} 关于连续性的定义)的,因为任意两个有理数之间必存在无理数(事实上,
对任意有理数$a$,~$b$,取$c=a+\frac{1}{\sqrt{2}}(b-a)$即可)。这说明有理数并不“完美”,直观上来说,
仅由有理数并不能生成一条连续的数轴,只能得到一系列离散的点——有理数与有理数之间是有“空隙”的。


而引入实数的概念后,连续性的问题得到了解决。我们是这样阐述实数的连续性的:对于集合$X,Y$,若$\forall x\in X
$,~$y\in Y$,~$x\leqslant y$,则$\exists c\in \mathbb{R}$,~$x\leqslant c\leqslant y$.可以这么说,无论两个实数多么“接近”,总有那么一个
实数可以“插入”到它们中间,这就是所谓“连续不断”。
\begin{example}
    取集合$X=\{\, x\mid x^2\leqslant 2\,\}$,~$Y=\{\, y\mid y^2>2\,\}$,显然满足$\forall x\in X,\,y\in Y$,~$x\leqslant y$.
    而$\exists ! c\in \mathbb{R}$,~$c^2=2$,~$x\leqslant c\leqslant y$.如果限制在有理数范围内,则无法找到这样的$c$.
\end{example}


上面几段文字的描述可能显得很抽象。但不要紧,多去找一些实例或推论,尝试自己去证明一些结论,理解必然会不断加深。
具体可以查询有关“实数完备性等价定理”的资料。
