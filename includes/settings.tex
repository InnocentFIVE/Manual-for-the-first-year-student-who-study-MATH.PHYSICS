% Sundry Settings
\setcounter{secnumdepth}{3}
\setlength{\lineskip}{5pt}
\setlength{\lineskiplimit}{2.5pt}
\linespread{1.4}
\let\left\mleft
\let\right\mright
\graphicspath{{./images/}}
\pgfplotsset{
    table/search path={./Plots},
}
\xeCJKsetup{CheckSingle = true}
\setlist[itemize, 2]{label = $\circ$}
\setlist[itemize]{listparindent = 2em}
\setlength{\marginparpush}{2em}
\setlength{\columnsep}{1cm}
\everymath{\displaystyle}
\newcommand{\link}[1]{\href{#1}{\faLink}}
\def\LaTeX{L\kern-.36em\raisebox{.38ex}{\textsc{a}}\kern -.16em T\kern -.1667em\raisebox{-.5ex}{E}\kern -.125em X}

% New theorem environments
\newtheorem{example}{\indent 例}
\theoremstyle{definition}
\newtheorem*{solve}{\indent 解}
\newtheorem*{prove}{\indent 证明}
\newtheorem*{glue}{\indent 思路}

% Color definition
\definecolor{macred}{HTML}{fc5450}
\definecolor{macyellow}{HTML}{e0a22d}
\definecolor{macgreen}{HTML}{24cd4c}
\definecolor{mmagray}{HTML}{d7d7d7}
\definecolor{EE6743}{HTML}{EE6743}

% Libraries of PKGs
\tcbuselibrary{breakable}
\tcbuselibrary{skins}
\tcbuselibrary{documentation}
\tcbset{breakable, enhanced jigsaw, fonttitle = \bfseries}
\usepgfplotslibrary{polar}
\usetikzlibrary{patterns}
\usetikzlibrary{shadings}
\usetikzlibrary {trees}
\usetikzlibrary {mindmap}

% Font Settings
\setsansfont{Noto Sans}
\setmonofont{Iosevka}[RawFeature=-calt; -dlig]
\newfontfamily\mdsans{Noto Sans Bold}
\setmathfont{texgyrepagella-math.otf}
\setmathfont{STIXTwoMath-Regular.otf}[range = {"1D49C-"1D4CF, "0212F-"02134, "0212C}]
\setmathfont{Fira Math}[version = fira]
\setmainfont{TeXGyrePagellaX-Regular}[
    BoldFont = TeXGyrePagellaX-Bold,
    ItalicFont = TeXGyrePagellaX-Italic,
    BoldItalicFont = TeXGyrePagellaX-BoldItalic,
    SlantedFont = TeXGyrePagellaX-Slanted]

% CJK *
\setCJKmainfont{Noto Serif CJK SC}[
    ItalicFont = FZKaiS-Extended,
    BoldItalicFont = FZKaiS-Extended,
]
\setCJKsansfont{Noto Sans CJK SC}[ItalicFont = FZKaiS-Extended]
\setCJKmonofont{Noto Sans CJK SC}[AutoFakeSlant]
\newCJKfontfamily\mdsansCJK{Noto Sans CJK SC Bold}
\newCJKfontfamily\itshapeCJK{FZKaiS-Extended-1.ttf}


\makeatletter
% CFT Settings
\setcounter{tocdepth}{1}
\newcommand\chapterbox[1]{%
    \begin{tcbox}[on line,
            outer arc = 2pt,
            colback = green!45!blue!50!black,
            boxsep = 0pt,
            left = 4pt,
            right = 4pt,
            top = 2.5pt,
            bottom = 2pt,
            boxrule = 0pt,
            arc = 2pt,
            colupper = white
        ]
        {#1}
    \end{tcbox}\hfill}
\def\cftdot{$\cdot$}
\def\addauthor#1{\addtocontents{toc}{\protect\vskip-17pt\protect\mbox{\hspace*{\textwidth - 13em}\itshape\color{gray}\footnotesize \underline{Author\,{\normalsize$|$}\footnotesize #1}}\par\bigskip}}
\renewcommand\cftchapfont{\bfseries}
\renewcommand\cftchapleader{}
\renewcommand\cftchapafterpnum{}
\renewcommand\cftchappagefont{\Large\bfseries}
\renewcommand\cftchapfillnum[1]{{\cftchapleader}{\makebox[\@pnumwidth][\cftpnumalign]{{\cftchappagefont #1}}\cftchapafterpnum\par}}
\renewcommand\cftchapfont[1]{%
    \begin{tcbox}[on line,
            outer arc=2pt,colback=green!45!blue!50!black,
            boxsep=0pt,left=4pt,right=4pt,top=2.5pt,bottom=2pt,
            boxrule=0pt,arc = 2pt,  colupper = white]
        {\bfseries #1}
    \end{tcbox}\hfill}
\renewcommand\cftsecindent{3.8em}
\renewcommand\cftsecfont{\color{black}\slshape}
\renewcommand\cftsecleader{\cftdotfill\cftdotsep}
\renewcommand\cftsecafterpnum{\hspace*{2em}\par\medskip}
\renewcommand\cftsecdotsep{0.05mu}
\renewcommand\cftsecpagefont{\itshape\sffamily}
\def\contentsname{\Huge 目录\\ \LARGE
    Contents%
    \bigskip\noindent}


% Mathematica
\newcounter{mma}
\setcounter{mma}{0}
\def\mathematicatitlename{八云蓝的挑战 - Wolfram Mathematica 13.1}
\def\mmain{\stepcounter{mma}\texttt{\color{gray}In[\arabic{mma}]:=}\ignorespaces}
\def\mmaout{\texttt{\color{gray}Out[\arabic{mma}]={ }}}
\newtcolorbox{notebook}[1]{
    colframe = lightgray!70!white,
    fontupper = \upshape\ttfamily,
    fontlower = \upshape\ttfamily,
    title = {
            \begin{tikzpicture}
                \filldraw[macred] (0, 0) circle (0.3em);
                \filldraw[macyellow!80!white] (0.5, 0) circle (0.3em);
                \filldraw[macgreen] (1, 0) circle (0.3em);
            \end{tikzpicture}%
            \hfill{\footnotesize\sffamily #1}\hfill%
            \phantom{
                \begin{tikzpicture}
                    \filldraw[macred] (0, 0) circle (0.3em);
                    \filldraw[macyellow!80!white] (0.5, 0) circle (0.3em);
                    \filldraw[macgreen] (1, 0) circle (0.3em);
                \end{tikzpicture}
            }
        }, coltitle = black, segmentation style = {draw = lightgray!70!white, very thick, solid}
}
\newenvironment{mathematica}{
    \begin{notebook}{\mathematicatitlename}%
        \def\key##1{{\itshape\color[HTML]{3c7d9e}##1}}%
        \def\UDsymbol##1{{\color[HTML]{002cc3}##1}}%
        \def\comment##1{{\itshape\color[HTML]{65b4cd}##1}}%
        \def\tab{\phantom{\ttfamily abcd}}%%
        \mmain%
        \def\^{{\char"005E}}%
        \def\@{{\char"0040}}%
        \def\~{{\char"007E}}%
        \def\ShiftEnter{\tcblower\mmaout\ignorespaces\def\^{{\char"005E}}%
        \def\@{{\char"0040}}%
        \def\~{{\char"007E}}}\space\ignorespaces
        }{\end{notebook}}


% listings Settings
\newcommand{\white}[1]{\makebox[2.5em][c]{\makebox[0pt][l]{\special{pdf:literal direct 2 Tr 0.3 w}%
            #1%
            \special{pdf:literal direct 0 Tr 0 w}}\raisebox{-0.09pt}{\includegraphics{white.pdf}}}}
\def\axisstyle#1{\itshape\color{MediumVioletRed}axis}
\lstset{
    basicstyle      = \ttfamily,
    language        = [latex]tex,
    commentstyle    = \itshape\color[HTML]{888888},
    columns         = fixed,
    basewidth       = 0.5em,
    alsoletter      = {*},
    keywordstyle    = \bfseries\color{Teal},
    texcsstyle      = *\color{CornflowerBlue!60!black}\bfseries,
    morekeywords    = {\documentclass,\usepackage,\begin,\end},
    emphstyle       = [1]\itshape\color{MediumVioletRed},
    emphstyle       = [2]\color{FireBrick},
    emphstyle       = [3]\bfseries\white,
    emphstyle       = [4]\bfseries\color{red},
    emphstyle       = [5]\axisstyle,
    emph            = {[1]equation,itemize,document,tikzpicture},
    emph            = {[2]article,amsmath,ctex},
    emph            = {[3]white},
    emph            = {[4]red},
    emph            = {[5]styleedaxis},
    frame           = trBL,
    backgroundcolor = \color{lightgray!30!white},
    keepspaces,
    breaklines,
    moretexcs={\colorbox,\color,\part,\subsection,\paragraph,\boldsymbol,\mathbb,\mathcal,\pgfplotsset,\addplot}
}

% Pagestyle / Title style Settings
\def\CTEX@part@pagestyle{empty}
\ctexset{
    section={
      name={第,节},
      number=\chinese{section},
      format=\Large\bfseries\raggedright
     },
    subsection={
            number = {(\chinese{subsection})},
        },
    subsubsection={
            number = {\Roman{subsubsection}},
            format = \mdsans\mdsansCJK
        }
}
\fancypagestyle{mainlatter}{
    \fancyfoot[C]{\thepage}
    \fancyhead[C]{
        \begin{tikzpicture}[remember picture,overlay]
            \node[opacity = 0.08] at (current page.center){
                \includegraphics[width = 0.8\textwidth]{LOGOs.png}
            };
        \end{tikzpicture}
    }
}

\csdef{abx@sstr@andothersincite}{等}
\NewDocumentEnvironment{marginparfigure}{+ b}{\marginpar{\centering\small#1}}{}
\NewDocumentEnvironment{marginpartext}{+ b}{\marginpar{\begin{abox}#1\end{abox}}}{}
\makeatother



% Autoref Settings
\def\subsubsectionautorefname{第\chinese{section}节--第(\chinese{subsection})节}
\def\subsectionautorefname{第\chinese{section}节}
\renewcommand{\figureautorefname}{图}
\renewcommand{\tableautorefname}{表}
\renewcommand{\equationautorefname}[2]{式}


% bibsettings
\renewcommand{\bibauthorfont}{\scshape}
\renewcommand{\bibtitlefont}{\bfseries}
\renewcommand{\bibpubfont}{\itshapeCJK\itshape}
\def\UrlFont{\ttfamily}

% Boxes
\newtcolorbox{questionbox}{
    colframe=black!60,% black frame colour
    width=15cm,% Use 5cm total width,
    arc=2mm, auto outer arc,
    boxrule=0.5pt,
    title=问题,
    fontupper = \itshape,
}
